%-------------------------------------------------
%	Version: 0.0
%	fecha de entrega 5 febrero 2021
%
%-------------------------------------------------

\documentclass[11pt]{report}

%packages
\usepackage{graphicx}
\usepackage{subcaption}

\usepackage[utf8]{inputenc}
\usepackage[spanish, es-nodecimaldot]{babel}
\usepackage{setspace}
\usepackage{ragged2e}

\usepackage{amsmath}
\usepackage{amsthm}
\usepackage{amssymb}
\usepackage{mathtools}
\usepackage{siunitx}
\usepackage[thinc]{esdiff} %derivadas faciles
\usepackage{physics} %algunos simbolos de derivadas

%path donde se encuentran las imagenes
\graphicspath{ {./figuras/} }

%---------------------------------------------------------------
%ABREVIACIONES DE COMANDOS
%---------------------------------------------------------------

\theoremstyle{plain}
\newtheorem{thm}{Teorema}[chapter] % reset theorem numbering for each chapter

\theoremstyle{definition}
\newtheorem{defn}[thm]{Definición} % definition numbers are dependent on theorem numbers
\newtheorem{exmp}[thm]{Ejemplo} % same for example numbers

\newcommand{\chaptercontent}{
\section{Basics}
\begin{defn}Here is a new definition.\end{defn}
\begin{thm}Here is a new theorem.\end{thm}
\begin{thm}Here is a new theorem.\end{thm}
\begin{exmp}Here is a good example.\end{exmp}
\subsection{Some tips}
\begin{defn}Here is a new definition.\end{defn}
\section{Advanced stuff}
\begin{defn}Here is a new definition.\end{defn}
\subsection{Warnings}
\begin{defn}Here is a new definition.\end{defn}
}

\usepackage{biblatex}
%\addbibresource{Tarea1.bib}

\begin{document}

\begin{titlepage}
\title{Titulo_del_trabajo}

%-------------------------------------------------
%PORTADA
%-------------------------------------------------

	\centering
	{\scshape\LARGE Universidad Autónoma de Yucatán  \\ Facultad de ingeniería\par}
	\vspace{1cm}
	{\scshape\Large Métodos matemáticos de la física\par}
	\vspace{1.5cm}
	{\huge\bfseries Segundo examen\par}
	\vspace{0.7cm}
	{\begin{figure}[!h]
	\centering
    \includegraphics[scale=0.3]{UADY.png}
	\end{figure}}
	\vspace{0.7cm}
	{\Large\itshape Erick Al. Casanova Cortés\par}
	{\Large\itshape Matricula: 15014866\par}
	\vfill
	{\scshape\Large Docente\par
	Dr. Miguel Zambrano\par}
	\vfill
	{\Large{\bfseries Fecha de entrega: 5 Febrero 2021} }

	\vfill
	
\end{titlepage}

%-------------------------------------------------
%Inicio del documento
%-------------------------------------------------

%-------------------------------------------------
%Primer Ejercicio
%-------------------------------------------------
\textit{Una esfera hueca de radio que esta dividida en dos por el ecuador por un aislante delgado. La mitad superior se encuentra a un potencial constante $V_0$ y el hemisferio inferior a un potencial nulo, como se muestra en la figura.}\\

\textit{Demostrar que el potencial dentro de la esfera esta dado por:}

\begin{equation}
	\phi(r, \theta) = \frac{V_0}{2}\{ 1 +\sum^\infty_{n=0}(-1)^n \frac{(2n)!(4n+3)}{2^{2n}(n!)(2n+2)}\left(\frac{r}{c}\right)^{2n+1}P_{2n+1}\cos(\theta)\}
\end{equation}

\textit{mientras que en el potencial exterior es:}
\begin{equation}
	\phi(r, \theta) = \frac{V_0c}{2r}\{ 1 +\sum^\infty_{n=0}(-1)^n \frac{(2n)!(4n+3)}{2^{2n}(n!)(2n+2)}\left(\frac{c}{r}\right)^{2n+1}P_{2n+1}\cos(\theta)\}
\end{equation}


\textit{Ayuda: resolver la ecuación de Laplace $\nabla^2 \phi(r,\theta,\varphi) =0$ en coordenadas esféricas}\\


%--- resolución de la ecuación de Laplace

Primero haremos el cálculo de la ecuación de Laplace en coordenadas esféricas, la cual está expresada de la forma

\begin{equation} %eq laplace en coordenadas esfericas
\label{eqn:laplace_spherical_coordinates}
	\left(\frac{1}{r^2}\pdv{r}r^2\pdv{r} + \frac{1}{r^2\sin\theta}\pdv{\theta}\sin\theta\pdv{\theta} + \frac{1}{r^2\sin^2\theta}\pdv[2]{\varphi}\right)F(r,\theta,\varphi) = 0
\end{equation}


De esto partimos que la función $F(r,\theta,\varphi)$ puede expresarse de tal forma que $F(r,\theta,\varphi)=R(r)\Theta(\theta)\Phi(\varphi)$. Por lo que podemos multiplicar ambas partes de la ecuación \ref{eqn:laplace_spherical_coordinates} por $\frac{r^2}{F}$, lo cual queda expresado como:

\begin{equation*} %multiplicando por r^2/f
	\frac{r^2}{F} \left(\frac{1}{r^2}\pdv{r}r^2\pdv{r} + \frac{1}{r^2\sin\theta}\pdv{\theta}\sin\theta\pdv{\theta} + \frac{1}{r^2\sin^2\theta}\pdv[2]{\varphi}\right)F(r,\theta,\varphi) = 0
\end{equation*}

Sustituyendo a $F(r,\theta,\varphi)$ por la propuesta


\begin{equation*} %sustituir por funciones dep
	\frac{r^2}{R\Theta\Phi} \left(\frac{1}{r^2}\pdv{r}r^2\pdv{r} + \frac{1}{r^2\sin\theta}\pdv{\theta}\sin\theta\pdv{\theta} + \frac{1}{r^2\sin^2\theta}\pdv[2]{\varphi}\right)R\Theta\Phi= 0
\end{equation*}

Ahora se expandirá la ecuación a conveniencia de tal modo que dejaremos los términos que dependen del radio de un lado y agruparemos los términos que dependen del ángulo polar o azimutal; quedando entonces:


\begin{equation*}%agrupar terminos
	\frac{1}{R}\pdv{r}r^2\pdv{r}R = -\frac{1}{\Theta\Phi}\left(\frac{1}{\sin\theta}\pdv{\theta}\sin\theta\pdv{\theta} + \frac{1}{\sin^2\theta}\pdv[2]{\varphi}\right)\Theta\Phi
\end{equation*}

Como $r$, $\theta$ y $\varphi$ son variables independientes quiere decir que ambos términos son iguales a una constante, lo que quiere decir que

\begin{equation}
\label{eqn:r_dep_sol_laplace} %parte de r
	\frac{1}{R}\pdv{r}r^2\pdv{r}R = \mathcal{C}
\end{equation}

\begin{equation}
\label{eqn:theta_phi_dep_sol_laplace} %parte de teta y phi
	-\frac{1}{\Theta\Phi}\left(\frac{1}{\sin\theta}\pdv{\theta}\sin\theta\pdv{\theta} + \frac{1}{\sin^2\theta}\pdv[2]{\varphi}\right)\Theta\Phi = \mathcal{C}
\end{equation}


si a la ecuación \ref{eqn:theta_phi_dep_sol_laplace} la multiplicamos por $\sin^2\theta$ tenemos:

\begin{equation*} % sin^2 tta
	-\frac{1}{\Theta\Phi}\left(\sin\theta\pdv{\theta}\sin\theta\pdv{\theta} + \pdv[2]{\varphi}\right)\Theta\Phi = \mathcal{C}\sin^2\theta
\end{equation*}

Agrupando la ecuación en sus términos dependientes nos queda la siguiente igualdad:

\begin{equation*} %agrupar
	\frac{1}{\Theta}\sin\theta\pdv{\theta}\sin\theta	\pdv{\theta}\Theta+\mathcal{C}\sin^2\theta = - \frac{1}{\Phi}\pdv[2]{\varphi}\Phi 
\end{equation*}

Como anteriormente se vio, para que esta igualdad se cumpla, ambas partes de dicha ecuación deberán ser igualadas a una constante. Lo cual queda de la siguiente forma:

\begin{equation}
	\label{eq:teta_sol}
	\frac{1}{\Theta}\sin\theta\pdv{\theta}\sin\theta	\pdv{\theta}\Theta+\mathcal{C}\sin^2\theta = \mathcal{B}
\end{equation}

\begin{equation}
	\label{eq:phi_sol}
	- \frac{1}{\Phi}\pdv[2]{\varphi}\Phi = \mathcal{B}
\end{equation}

Haciendo un despeje de la ecuación \ref{eq:phi_sol} podemos ver que podemos encontrar una solución, como se ve a continuación

\begin{equation*}
	\pdv[2]{\varphi}\Phi = -\Phi\mathcal{B}
\end{equation*}

al resolver la ecuación diferencial podemos ver que $\Phi$ está dada por:

\begin{equation*}
	\Phi(\varphi) = e^{\pm i\sqrt{\mathcal{B}}\varphi}
\end{equation*}

Si hacemos un análisis de esta ecuación podemos ver que es periódica con un ciclo de $2\pi$, lo que implica:

\begin{equation*}
	e^{\pm i\sqrt{\mathcal{B}}\varphi} = e^{\pm i\sqrt{\mathcal{B}}(\varphi+2\pi)}
\end{equation*}

En otras palabras, lo que quiere decir es que $\sqrt{\mathcal{B}}$ debe ser un entero, y por fines prácticos haremos la igualdad $\sqrt{\mathcal{B}} = k^2$, regresando a la solución tenemos entonces que la parte de $\Phi$ queda expresada como:

\begin{equation*}
	\Phi(\varphi) = e^{\pm ik^2\varphi}
\end{equation*}

Pasando ahora a la parte que depende de $\Theta$, partimos de la ecuación \ref{eq:teta_sol} y tomando la igualdad que encontramos anterior, la cual nos afirma que $\sqrt{\mathcal{B}} = k^2$

\begin{equation*}
	\frac{1}{\Theta}\sin\theta\pdv{\theta}\sin\theta	\pdv{\theta}\Theta+\mathcal{C}\sin^2\theta = k^2
\end{equation*}

Para hacer resolver esta ecuación diferencial tendremos que hacer un cambio de variable, el cual será proponer una función a modo que:

\begin{equation*}
	x=\cos\theta
\end{equation*}

Haciendo la regla de la cadena podemos ver:

\begin{align*}
	\pdv{\theta} &= \pdv{x}{\theta}\pdv{x}\\
	&=-\sin\theta \pdv{x}
\end{align*}

Por otra parte tenemos la igualdad

\begin{equation*}
	\sin^2\theta = 1-\cos^2\theta
\end{equation*}

Por lo que con nuestro cambio de variable podemos ver que $\sin^2\theta = 1-x^2$. Sustituyendo en la ecuación podemos ver que:

\begin{equation*}

	\frac{1}{\Theta}\sin\theta\pdv{\theta}\sin\theta	\pdv{\theta}\Theta+\mathcal{C}\sin^2\theta = k^2
\end{equation*}
%-------------------------------------------------
%Final del documento
%-------------------------------------------------

\end{document}
